\pdfsuppresswarningpagegroup=1
\pdfminorversion=7
% StM3 Assessed Exercise
% Template by: Dr. Antonio Melro <antonio.melro@bristol.ac.uk>
% Private use only
% Last modification: 11/12/2018$
%~~~~~~~~~~~~~~~~~~~~~~~~~~~~~~~~~~~~~~~~~~~~~~~~~~~~~~~~~~~~~
% Document size & margin definitions
\documentclass[11pt,a4paper,oneside]{memoir}
%~~~~~~~~~~~~~~~~~~~~~~~~~~~~~~~~~~~~~~~~~~~~~~~~~~~~~~~~~~~~~
%\counterwithout{section}{chapter}
\setsecnumdepth{subsection}
\setsecnumdepth{subsubsection}
% Latex workpackages
\usepackage{amsmath}
\usepackage{amssymb}
\usepackage{graphicx}
\graphicspath{ {./figs/}}
\usepackage{mathpazo}
\usepackage{microtype} % Slightly tweak font spacing for aesthetics
\usepackage[pdfpagelabels]{hyperref}
\usepackage{booktabs}
\usepackage{multirow}
\usepackage{tikz}
\usepackage{xcolor}
\usepackage{framed}
\usepackage{geometry}
\newgeometry{innermargin=1.25in,top=1.25in,bottom=1in}
%\usepackage{caption}
\usepackage[tikz]{bclogo}
\usepackage{lipsum}
\hypersetup{
    colorlinks,
    citecolor=blue,
    filecolor=blue,
    linkcolor=blue,
    urlcolor=blue
}
%
\usepackage{textcase}
\newcommand{\theauthor}{\href{dp16812@my.bristol.ac.uk}{Dharapa Pattanakul}}
\newcommand{\studentid}{dp16812}
\newcommand{\tutor}{\href{brano.titurus@bristol.ac.uk}{Brano Titurus}}
\newcommand{\coursetitle}{Structures \& Materials 3}
\newcommand{\thetitle}{Metallic Wing Torque Box\\Assignment Report}
\newcommand{\subtitle}{\sffamily\textls[200]{\MakeTextUppercase{Stress Analysis using MSC Patran/Nastran}}\normalfont}
\newcommand{\department}{Department of Aerospace Engineering \vspace{0.35cm}\par University of Bristol}
%
\newcommand{\monthyear}{\ifcase\month\or January\or February\or March\or April\or May\or June\or July\or August\or September\or October\or November\or December\fi\space\number\year} % A command to print the current month and year


% Begin document
\begin{document}
\pagenumbering{Alph}
% Title page
%\frontmatter

\begingroup
\thispagestyle{empty}
\newgeometry{innermargin=1in,top=1.0in,bottom=1in}

\setlength{\parindent}{0pt}
%\begin{minipage}[b]{0.5\linewidth}
%\vfill\fontsize{14}{14}\selectfont\textit{\department} 
%\end{minipage}
%\hspace{0.5cm}
%\begin{minipage}[b]{0.45\linewidth}
%\hfill\includegraphics[width=6.0cm]{UoB-logo-black.pdf}
%\end{minipage}
\begin{minipage}[b][2.0cm]{0.5\linewidth}
\vfill\fontsize{14}{14}\selectfont\textit{\department} 
\end{minipage}
\begin{minipage}[b][2.0cm]{0.45\linewidth}
\hfill\includegraphics[width=6.0cm]{UoB-logo-black.pdf}
\end{minipage}

\vspace{1.75in}\fontsize{36}{54}\selectfont\coursetitle

\vspace{.75in}\fontsize{36}{54}\selectfont\thetitle

\vspace{0.125in}\fontsize{14}{14}\selectfont\subtitle
\vfill
\begin{minipage}{0.48\textwidth}
\fontsize{16}{24}\selectfont\textsc{Student:}

\fontsize{16}{24}\selectfont\textsc{Student Number:}

\fontsize{16}{24}\selectfont\textsc{Tutor(s):}

\fontsize{16}{24}\selectfont\textsc{Academic Year:}
\end{minipage}
\begin{minipage}{0.48\textwidth}
\fontsize{16}{24}\selectfont\textnormal{\theauthor}

\fontsize{16}{24}\selectfont\textnormal{\studentid}

\fontsize{16}{24}\selectfont\textnormal{\tutor}

\fontsize{16}{24}\selectfont\textnormal{2018-19}
\end{minipage}

\restoregeometry
\endgroup

% abstract page
\newpage
~\vfill
\thispagestyle{empty}
\setlength{\parindent}{0pt}
\setlength{\parskip}{\baselineskip}
\large Academic Year \the\year-19 \par
\large Structures \& Materials 3 (AENG31200) \par
\large Academic Lead(s): \, \href{giuliano.allegri@bristol.ac.uk}{Dr.~Giuliano Allegri}, \href{antonio.melro@bristol.ac.uk}{Dr.~Ant\'{o}nio Rui Melro} \par
\par Department of Aerospace Engineering\index{license}

\par\textit{University of Bristol, \monthyear}
\cleardoublepage

\frontmatter
% Frontpage:
\renewcommand{\abstractname}{Summary}
\begin{abstract}
	%Summary should be written in simple language (avoiding core technical terms) clearly stating the objective of the analysis, main conclusions and recommendations. It is meant for managers, decision makers and team members involved in the project who either do not have sufficient time to go through the complete report or are not familiar with FEA terminology.
	The finite element model of the wing torque box was modelled and analysed using the software MSC PATRAN and NASTRAN . The model was compared to the analytical calculation and the results concluded that the FE model can reliably performs modelling analysis for this case. The difference in the results was acceptable as the deflection of 6 mm is small relative to the size of the wing. The model with inspection holes in the bottom skin does not affects the deflection calculated using FE in a scale that needs to be accounted for.\\ \\
	The recommendations for improvements include using a finer mesh for FE model and accounting for nonlinear behaviours in load case, material properties and deformations to represent the problem more accurately.
\end{abstract}

\clearpage
\tableofcontents% automatically create table of contents at second Latex run
\newpage
\listoffigures
\listoftables
\clearpage % start new page

\mainmatter
\chapterstyle{reparticle}
\pagestyle{ruled}

%\chapter{Format of technical report}

%The final report will be submitted online via SAFE as a PDF, no hardcopy necessary!
%A report template is available on Blackboard. You may use either Microsoft Word, or \LaTeX. Technical report should include:
	%\begin{enumerate}
	%	\item \textbf{Title or front page of report:} The title of the project, a nice small figure of the component, a report number, date of submission, name of supervisor, name of analyst (student), student contact details and affiliation.
	%	\item \textbf{Summary (maximum 1 page) of the project:} A summary should be written in simple language (avoiding core technical terms) clearly stating the objective of the analysis, main conclusions and recommendations. It is meant for managers, decision makers and team members involved in the project who either do not have sufficient time to go through the complete report or are not familiar with FEA terminology.
		%\item \textbf{Main body of work:}
		%\begin{enumerate}
		%	\item Aims/Scope of project.
		%	\item Brief description of component, basic design details, functionality.
		%	\item Methodology or strategy of analysis.
		%	\item Mesh details (element types used, number of elements etc.), quality checks.
		%	\item Material and sectional properties.
		%	\item Boundary conditions details with separate figure corresponding to the appropriate load case.
		%	\item Tabular results and clear figures of appropriate load case.
		%\end{enumerate}
		%\item \textbf{Conclusion and future work}
	%\end{enumerate}

%Documentation and local data storage/backup of all relevant models and results are essential.  

\chapter{Introduction to StM3 Assignment}\label{chap:intro}
\section{Aims/Scope of project}
\begin{enumerate}
	\item This exercise was carried out to gain a better understanding of finite element analysis modelling and to investigate the differences between analytical results and finite element results. 
	\item The model was compared to the AVDASI 2 wing project calculation. The real model of this torque box was made as a part of AVDASI 2 unit and by carrying out this exercises and compares the computational results, analyticl results and the actual test that was carried out in AVDASI 2, this provides  good insight in to differences and limitations of the theoretical prediction and the real world experiments.
\end{enumerate}
%
%\begin{quote}
%...the aim of this assignment is for you to broaden your FE skills and have practical experience tackling a \textbf{real-world exercise} by using an aerospace industry-standard, FE software package; MSC Patran/Nastran. A useful reference is a peer-reviewed journal paper by Ostegaard et al.~on the application of virtual testing of aircraft structures using non-linear finite element models \cite{ostergaard2011virtual}...

%...to realise this goal, a finite element model of a wing torque box is developed based on a generic design from AVDASI 2, see Fig.~\ref{fig:ASD2-DBT-Wing-5-revised}, and perform a linear static stress analysis to assess its structural performance (by calculating deflections and stresses)...
%\end{quote}

\begin{figure}
  \includegraphics[width=\textwidth]{ASD2-DBT-Wing-5-revised-2018.pdf}
  \caption{Schematic of a typical wing torque box - NACA 2418 profile. (LE) - Leading edge, (TB) - Torque box, (TE) - trailing edge  - \textit{(Ian Farrow 2013)}.}
  \label{fig:ASD2-DBT-Wing-5-revised}
\end{figure}

\section{Description of wing torque box}

\begin{enumerate}
	\item The wing torque box as highlighted in blue is made of aluminium sheet. The box is a long hollow and slender body with a rib every 250 mm and L-shape angle supports in 4 corners.  
\end{enumerate}
%

\begin{quote} A CAD model of a generic torque box (TB) prototype is shown in Fig.~\ref{fig:wing-cad-model}. Key structural features are: 

\begin{itemize}
\item \textbf{Spars} Vertical elements running along the wingspan. They have the main function of carrying the shear force; 
\item \textbf{Stringers} Slender beams attached to the skin. They have the main function of carrying the axial force (thus balancing the wing bending moment) and stabilizing the skin against buckling; 
\item \textbf{Skin} Thin shell covering the wing. It has the main function of carrying the wing torsional moments and providing the aerodynamic shape; 
\item \textbf{Ribs} Elements located in the cross-section plane. They have the main function of maintaining the shape and redistributing the external loads.
\end{itemize}
%
Although several simplifications have been made (e.g.~absence of LE droop-nose and TE flap), the proposed structure contains sufficient detail for an initial FE analysis to capture the phenomena of interest. The CAD that has been prepared for this exercise is a reduced version of this wing torque box, containing only surfaces and curves (lines) to represent the main geometric features (see Fig.~\ref{fig:wing-cad-model-cleanup})...\end{quote}
%
\begin{figure}
  \begin{center}
  \includegraphics[width=\textwidth]{wing-cad-model-2018.pdf}
  \caption{Representative CAD model of torque wing box highlighting key structural features.}
  \label{fig:wing-cad-model}
  \end{center}
\end{figure}
%
\begin{figure}[t]
  \begin{center}
  \includegraphics[width=1\textwidth]{wing-cad-model-cleanup.pdf}
  \caption{Final IGES file available on Blackboard after CAD clean-up and repair.}
  \label{fig:wing-cad-model-cleanup}
  \end{center}
\end{figure}

\subsection{Boundary and loading conditions} 
\label{BD}
The torque box is idealised as a cantilever beam with an offset load, as illustrated in Fig. \ref{fig:ASD2-DBT-Wing-TB-Loading-Boundary}. Only one load case shall be considered, where the value of \mbox{$P = 1.0$ kN}. 
To best mimic the structural root attachments from AVDASI2, the following boundary conditions (BCs) are suggested:
\begin{description}
\item[1. Trailing Edge spar:] To represent the TE bending reaction joint plates, assign fully constrained boundary conditions at the TE, see Fig.~\ref{fig:ASD2-DBT-Wing-TB-Loading-Boundary}. 	
\item[2. Baseplate:] Entire wing root is constrained in the global $X$ (span-wise) direction. Note that this BC implies that all of the sections are active at the root.
\end{description}

\begin{figure}
    \centering
    \includegraphics[width=.98\textwidth]{ASD2-DBT-Wing-TB-Loading-Boundary-2018}
    \caption{Schematic of boundary (restraint) and loading conditions applied to wing torque box \textit{(Ian Farrow 2013)}.}
  \label{fig:ASD2-DBT-Wing-TB-Loading-Boundary}
\end{figure}
%\begin{figure}
%    \centering
%    \includegraphics[scale=1.0]{ASD2-DBT-Wing-TB-Loading-Boundary-6}
%    \caption{Schematic of structural test arrangement of root fixture boundary conditions \textit{(Ian Farrow 2013)}.}
%  \label{fig:ASD2-DBT-Wing-TB-Loading-Boundary-2}
%\end{figure}
\clearpage
\subsection{Geometric properties of structural elements}

Table \ref{tab:wing-size} gives the dimensions of key structural elements of the torque box that should be employed in the sectional properties definition of the finite element model, as illustrated in Fig.~\ref{fig:ASD2-DBT-Wing-5-revised}.

%\begin{figure}
    %\begin{center}
    %\includegraphics[scale=1.0]{wing-cad-model-view.pdf}
    %\caption{Cross-section of the wing torque box showing key structural features: skin, stringers and spars.}
    %\label{fig:wing-cad-model-view}
    %\end{center}
%\end{figure}

Spar cap angles are composed of L profiles cross-sectional areas. 

All structural elements are manufactured from the aerospace standard clad aluminium alloy 'Aluminium 2014a-T3'\footnote{Recall temper notation : T3 - Solution heat treated, cold worked and naturally aged}. Typical material properties for 2014 aluminium alloy are given in Table \ref{tab:wing-size}.  
%
\begin{table}[!h]
\centering
\caption{Dimensions and sizes of the wing and its structural elements.}
\label{tab:wing-size}
\begin{tabular}{llc}
\toprule
Element              & Thickness [mm]  &   Additional comments  \\    
\midrule
UPR skin             & 0.8         &  \\
LWR skin             & 0.5         &  \\
                     &             &  \\
LWR spar cap angle   & 0.9         &  All angles: L profile  \\
UPR spar cap angle   & 0.7         &  15 $\times$ 15 mm\\
                     &             &  \\
LE spar              & 0.5         &  \\
TE spar              & 0.5         &  \\
                     &             &  \\
Ribs                 & 1.0         &  \\
                     &             &  \\
%Span length          & 1500        &  \\
\midrule
\multicolumn{3}{c}{Material : Aluminium 2014a-T3}\\
\multicolumn{3}{c}{E = 73 GPa, $\nu$ = 0.3, $\rho$=2800 kg/m$^{3}$}\\
\multicolumn{3}{c}{Tensile and compressive allowable limit: 245 MPa}\\
\bottomrule
\end{tabular}
\end{table}

%
\chapter{Finite Element Model}\label{chap:fe-model}
\section{Pre-processing steps}Discuss very briefly the pre-processing steps:
\begin{enumerate}
    \item The IGE file which contains the torque box geometry was imported into PATRAN.The surfaces were then broken along the span into six equal sections. This allows the ribs to be created. The surfaces were broken down by the projection of the inboard edges onto 5 equally spaced planes.
    \item The unit system used in this model was millimetre for length and Newton for force.
    \item Material used in this model was Aluminium 2014a T3 with the properties as stated in table \ref{tab:wing-size}. The model used assumed an isotropic material which means the material is arranged in the same direction for the whole torque box model. Material decription is shown in figure \ref{fig:ex.1-bdf} labeled \textbf{4.}
    \item The skins of the box were assigned a 2D shell properties with varied thickness as indicated in table \ref{tab:wing-size}. This can also be seen in figure \ref{fig:ex.1-bdf} where \textbf{3.} shows different thickness for different shell properties. The spar angle supports were created using the 1D beam element properties with 2 different thickness for upper and lower skins. They also have different orientation for 4 corners. The properties of these beams is shown in figure \ref{fig:ex.1-bdf} where \textbf{1.} shows the beam dimension in mm. and \textbf{2.} shows beam's thickness.
    \item The conditions stated in figure \ref{fig:ASD2-DBT-Wing-TB-Loading-Boundary} means the conditioned applied is 
    \begin{itemize}
        \item fixed in all 6 degrees of freedom at the trailing edge spar.
        \item fixed in translational x direction for the rest of the wing.
        \begin{figure}[h]
            \centering
            \includegraphics[width = .5\textwidth]{figures/boundary-condition.png}
            \caption{Boundary condition as applied in PATRAN}
            \label{fig:ex.1-bd}
        \end{figure}
    \end{itemize}
    PATRAN boundary condition constraints are shown in figure \ref{fig:ex.1-bd} .\\
    The load property  as listed in the .bdf file is labeled \textbf{5.} in figure \ref{fig:ex.1-bdf}. 

\begin{figure}[h]
    \centering
    \includegraphics[width = .9\textwidth]{figures/bdf.png}
    \caption{Parts of the .bdf file which shows the shell and beam properties as well and boundary conditions and load.}
    \label{fig:ex.1-bdf}
\end{figure}

    \item The mesh on the skin (this included ribs, spars, upper and lower skin.) were meshed using CQUAD4 elements as they are 2D shell. The rectangular shape mesh was suitable in this case because of the load case and the boundary conditions. The torque box was fixed at the root and only one load acted at a point load in y direction. The 1D CBEAM element was used for the angle caps to model the L-shape support on 4 corners.  Number of element was chosen so that the load can be put accurate onto a node and that the mesh is fine enough to represent a reliable model. The number of elements and their aspect ratio can be seen in  Table \ref{tab:summary-mesh}.
\end{enumerate}

\begin{table}[h]
\centering

\caption{Summary of element types and aspect ratios assigned to each structural entity}
\label{tab:summary-mesh}	
\begin{tabular}{lccccc}
\toprule
Part               & \multicolumn{4}{c}{Total No. of elements}   & Element      \\
\cmidrule{2-5}     &  CBAR & CBEAM  & CQUAD4  & CTRIA3           &   size (AR)  \\
\midrule
UPR/LWR Skin       &   -    &  -      & 6900        &  -                &       2       \\
TE/LE Spar webs    &    -   &   -     &  2400       &     -             &        1.075      \\
TE/LE Angle caps   &  -     &   600     &    -     &    -              &        N/A      \\
Ribs               &   -    &     -   &  840       &        -          &           2   \\
\midrule
Total              &    -   &    600    &     10140    &       -           &              \\
\bottomrule
\end{tabular}
\end{table}

\chapter{Analytical Wing Sizing Calculations}\label{chap:hand-calcs}
\section{Initial wing sizing calculations}
	%\item You may tabulate the results directly from a working excel spreadsheet or Matlab script. However, place key calculations in the Appendix not in the main body of text. 
\begin{enumerate}
    \item Key assumptions,
    \begin{itemize}
        \item The structure is modelled using the thin-wall assumptions for semi-monocoque structure.
        \item Also, assume the skin carries only shear when buckled.
        \item  Assume only the skin area get stabilised by the corner flanges when estimate the effective end load.
        \item Reduce the stabilised web and angle blade areas by one third to avoid overestimating the second moment of area contribution of the spar web and angle blade to the box.
    \end{itemize}
    \item The equivalent torque box is shown in figure \ref{fig:box}. The idealisation booms are also shown in figure \ref{fig:UPR-boom} and figure \ref{fig:LWP-boom}. 
    \begin{figure}[h]
        \centering
        \includegraphics[width = .7\textwidth]{figures/box.png}
        \caption{equivalent torque box}
        \label{fig:box}
    \end{figure}
    \begin{figure}[!hbt]
    \begin{minipage}{.5\textwidth}
    \centering
    \includegraphics[width = \textwidth]{figures/boom-1.png}
    \caption{Upper boom idealised}
    \label{fig:UPR-boom}
    \end{minipage}%
    \begin{minipage}{.5\textwidth}
    \centering
    \includegraphics[width = \textwidth]{figures/boom-2.png}
    \caption{Lower boom idealised}
    \label{fig:LWP-boom}
    \end{minipage}
\end{figure}
\end{enumerate}

\chapter{Results of Finite Element Model}\label{chap:results}
\section{Results from Exercise \#1}
\begin{enumerate}
	\item Table \ref{tab:summary-results} shows important results for the load case described in section 1.2.1.
\end{enumerate}

\begin{table}[h]
\centering
\caption{Summary of stress and deflection results as obtained from MSC Nastran linear static model.}
\label{tab:summary-results}
\begin{tabular}{lccc}
  \toprule
  Component                           & Value & Location & Fig. No.   \\
  \midrule
  Max. displacement                   &  20.2 mm. &     Node 4679     &     \ref{fig:max-displacement}       \\
  \midrule
  Max. von Mises stress (shell)       &  132 Nmm\textsuperscript{-2}& Node 313  &    \ref{fig:max-von-mosses}        \\
  \midrule
  Max. shear stress (shell)           &  71.2 Nmm\textsuperscript{-2} & Node 313 &   \ref{fig:max-shear}        \\
  \midrule
  Max. combined axial and             &  208 Nmm\textsuperscript{-2} & Node 313 &   \ref{fig:combined-beam}         \\
  bending stress (beam)               &       &          &            \\
  \midrule
  Mass of the wing torque box         &  1.537 kg &    n/a   &   n/a      \\  \bottomrule
\end{tabular}\end{table}
\begin{figure}[h]
    %\begin{minipage}{.5\textwidth}
    \centering
    \includegraphics[width = .8\textwidth]{figures/one-displacement-2.png}
    \caption{Max. Displacement of the whole torque box}
    \label{fig:max-displacement}
    %\end{minipage}%
\end{figure}
\begin{figure}[h]
    %\begin{minipage}{.5\textwidth}
    \centering
    \includegraphics[width = .8\textwidth]{figures/one-von-misses.png}
    \caption{Max. von Misses stress}
    \label{fig:max-von-mosses}
    %\end{minipage}
\end{figure}
\begin{figure}[h]
    %\begin{minipage}{.5\textwidth}
    \centering
    \includegraphics[width = .8\textwidth]{figures/one-max-shear.png}
    \caption{Max. shear stress}
    \label{fig:max-shear}
    %\end{minipage}%
\end{figure}
\begin{figure}[h]
    %\begin{minipage}{.5\textwidth}
    \centering
    \includegraphics[width = .8\textwidth]{figures/one-combined-beam-stress.png}
    \caption{Max. combined axial and bending stress in beam elements.}
    \label{fig:combined-beam}
    %\end{minipage}
\end{figure}

\section{Verification against Analytical Solutions}
\begin{enumerate}
	\item Compare and discuss analytical and FE reserve factors.
\end{enumerate}

\begin{center}
	\begin{tabular}{ccccccc}
		\toprule
		            & UPR TE   & UPR skin & TE spar  & LWR skin & LWR TE  & Max tip  \\
		            & col. bkl.& buckling & buckling & tension  & tension & deflect. \\
		\midrule
		FE          &  1.266  &  0.622  &  0.823  &  2.616 & 1.533  &  20.2 mm   \\
		\midrule
		Analytical  &  1.316  &  0.798  &  0.861  &  3.707  & 1.777  &  26.2 mm   \\
		\midrule
		Difference & 4\% & 28\% & 5\% & 42\% &16\% & 29\% \\
		\bottomrule
	\end{tabular}
\end{center}

\section{Discussion of Results}
%\begin{enumerate}
	%\item Use engineering judgement to comment on the structural integrity of the wing torque box based on your stress results and stress allowable for the aluminium material. 
	%\item Does the FE model fully capture the structural behaviour of the wing torque-box under the loading specified?
	%\item What are the limitations of the current 'linear static model'?
%\end{enumerate}

\begin{enumerate}
    \item The maximum deflection predicted of 20.2 mm be the finite element model and 26.2 mm predicted by the analytical result in acceptable. The deflection is small the for load of 1000 kN. 
    \item The FE model yielded lower reserve factors for every parameters calculated. This can be seen clearly with the lower predicted deflection.The differences for most parameters except the lower skin tension reserve factor are small. This shows that the FE model does capture the structural behaviour as it agrees with the analytical solutions. The differences can be accounted for for the limitations of the linear static model.
    \item The limitations of the current linear static model include,
    \begin{itemize}
        \item Geometry nonlinearities - if the box buckles and becomes no longer linear, the model cannot accurately predict the behaviour of the box under load.
        \item No vibration - the model assume no vibrational response to the load applied which is not true in real world. 
        \item The model assume a linear material which accounts to some failure in read world case.
        \item Only works for a static case as the model does not include inertia.
    \item However, by comparing the results from FE to those of the analytical ones, it can be seen that the FE model is good enough. The linear static model is convenient because it is fast, is available in abundant, and is well understood as it has been in use for many years.
    \end{itemize}
\end{enumerate}

\section{Improvements to FE Design - Inspection hole analysis}
\begin{enumerate}
	%\item Present images of the von Mises stress field on the lower skin without and with inspection holes. Discuss differences.
	\item Calculate the stress concentration factor on each inspection hole. Present results in a table.
	\item Present a graph showing the distribution of stress in a chord-wise direction on an inspection hole. Discuss stress concentration around a hole.
	\item Discuss the re-distribution of stress after including the inspection holes. How has the stress been distributed in the skin/booms?
\end{enumerate}

\begin{enumerate}
    \item The geometry of the inspection holes were created by the following steps:
    \begin{itemize}
        \item Six planes were created in the middle of the six bays with the vector [0,1,0] as the reference vector. A circle with radius of 15mm was then created on each plane. Each lower skin panel was then being sectioned as shown in figure \ref{fig:ex.3-surface}
        \begin{figure}[h]
            \centering
            \includegraphics[width = .7\textwidth]{figures/LWR-surface.png}
            \caption{Surface break of a section from a lower skin panel.}
            \label{fig:ex.3-surface}
        \end{figure}
        The surfaces were created  in this panel apart from inside the circle; this created a hole.
    \end{itemize}
    \item Figure \ref{fig:von-misses-holes} and \ref{fig:von-misses-w/o-holes} show the von Misses stress distribution along the lower skin in the case with and without the inspection holes. Without the hole, the max. stress is on the root corner furthest from where the load was applied. However, when the hole is present, the max. stress is around the hole. 
    \begin{figure}[h]
        \centering
        \includegraphics[width = .8\textwidth]{figures/stress-distribution.png}
        \caption{von Misses stress distribution in the lower skin with inspection holes.}
        \label{fig:von-misses-holes}
    \end{figure}
    \begin{figure}[h]
        \centering
        \includegraphics[width = .8\textwidth]{figures/one-LWR-skin-von-misses.png}
        \caption{von Misses stress distribution in the lower skin without inspection holes.}
        \label{fig:von-misses-w/o-holes}
    \end{figure}
\begin{figure}[!hbt]
    \begin{minipage}{.5\textwidth}
    \centering
    \includegraphics[width = .92\textwidth]{figures/three.png}
    \caption{A closer look at figure \ref{fig:von-misses-holes}}
    \label{fig:von-misses-holes-zoom}
    \end{minipage}%
    \begin{minipage}{.5\textwidth}
    \centering
    \includegraphics[width = .95\textwidth]{figures/one.png}
    \caption{A closer look at figure \ref{fig:von-misses-w/o-holes}}
    \label{fig:von-misses-w/o-holes-zoom}
    \end{minipage}
\end{figure}
    \textbf{DISCUSS DIFFERENCES}
    \item Table \ref{tab:stress-conc-factor} shows stress concentration factor in each inspection hole. The table which shows calculation for this can be found in appendix \textbf{APPENDIX IT}
    \item Figure \ref{fig:stress-along-chord-hole} and figure \ref{fig:stress-along-chord} show the stress distribution in a chord-wise direction. Figure \ref{fig:stress-along-chord-hole} shows the distribution on an inspection hole while figure \ref{fig:stress-along-chord} shows the one without. The illustration of these stress cases can be seen in figure \ref{fig:von-misses-holes-zoom} and figure \ref{fig:von-misses-w/o-holes-zoom}. The two low spikes can be seen on both side of the hole just before the spar attachment and the highest peak represents the maximum stress in the case with the inspection hole. In the case without the inspection hole, the stress distribution does not show a big change and the stress is distributed evenly along the sheet.
\end{enumerate}

\begin{table}[h]
    \centering
    \begin{tabular}{l c}
    \toprule
      Hole   &  Stress concentration factor\\
     \midrule
      1 (250mm)  & 2.34\\
      2 (500 mm) &   3.12\\
      3 (750 mm) &   2.62\\
      4 (1000 mm) &   2.74\\
      5 (1250 mm) &   5.47\\
      6 (1500 mm) &   1.72\\
      \bottomrule
    \end{tabular}
    \caption{Stress concentration factor in each inspection hole.}
    \label{tab:stress-conc-factor}
\end{table}

\begin{figure}[h]
    \centering
    \includegraphics[width = .8\textwidth]{figures/Path-length.png}
    \caption{Distribution of stress along the chord-wise direction with an inspection hole.}
    \label{fig:stress-along-chord-hole}
\end{figure}

\begin{figure}[h]
    \centering
    \includegraphics[width = .8\textwidth]{figures/Path-length-one.png}
    \caption{Distribution of stress along the chord-wise direction.}
    \label{fig:stress-along-chord}
\end{figure}



\chapter{Conclusions and Future Work}

\bibliographystyle{unsrt} % Title is link if provided
\bibliography{bibliography} % adjust this to fit your BibTex file

%: ----------------------- appendices ------------------------
\appendix

\chapter{MSC Nastran .bdf file}\label{chap:a}
\begin{enumerate}
	\item Copy of \underline{\textbf{truncated}} .bdf file from Exercise \#1.
\end{enumerate}

\chapter{Analytical Calculations}\label{chap:b}

% Backmatter

\pagestyle{empty}
\newgeometry{innermargin=1in,top=1.25in,bottom=1.25in}
\vspace*{17.5cm}
\begin{minipage}[b]{14cm}
          \begin{tabular}{l}
          \includegraphics[width=6.0cm]{UoB-logo-black.pdf}\\
            \Large{\url{www.bristol.ac.uk/engineering/}}\\
                \large Department of Aerospace Engineering \\
                \large University of Bristol\\
                \large Queen's Building \\
                \large University Walk\\
                \large Bristol, BS8 1TR \\
                \large United Kingdom\\
            \hspace*{-0.2cm}\large \begin{tabular}{l l l}
               Tel:& \quad   &(+44) (0) 117 33 15311\\
               Fax:& \quad   &(+44) (0) 117 954 5666\\
%\large Email: & \quad & \href{mailto:antonio.melro@bristol.ac.uk}{antonio.melro@bristol.ac.uk}\\
            \end{tabular}\\         
            \\
            \\
          \end{tabular}
        \end{minipage}
\restoregeometry
\end{document}

\end{document}
%~~~~~~~~~~~~~~~~~~~~~~~~~~~~~~~~~~~~~~~~~~~~~~~~~~~~~~~~~~~~~
\end{document}  % End of document (closes \begin{document} command).
%~~~~~~~~~~~~~~~~~~~~~~~~~~~~~~~~~~~~~~~~~~~~~~~~~~~~~~~~~~~~~
